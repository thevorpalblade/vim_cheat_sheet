\documentclass{article}
\usepackage [english]{babel}
\usepackage [autostyle, english = american]{csquotes}
\MakeOuterQuote{"}

\begin{document}
\section{General}

    Nearly all commands can be preceded by a number for a repeat count. eg. 5dd delete 5 lines
    $<$Esc$>$ gets you out of any mode and back to command mode
    Commands preceded by : are executed on the command line at the bottom of the screen
    :help help with any command

\section{Navigation}

    \subsection{By words:}
        \begin{itemize}
        \item w next word (by punctuation); W next Word (by spaces)
        \item b back word (by punctuation); B back Word (by spaces)
        \item e end word (by punctuation); E end Word (by spaces)
        \end{itemize}
    \begin{itemize}
    \subsection{By line:}
        \item 0 start of line; \textasciicircum  first non-whitespace
        \item \$ end of line
        \end{itemize}
    \subsection{By Sentence}
    Parenthesis!\\
                (: to the end of the next sentence\\
                ): to the beginning of the next sentence.
        
    \subsection{By paragraph:}
        \begin{itemize}
        \item \{ previous blank line; 
        \item \} next blank line
        \end{itemize}
    \subsection{By file:}
        \begin{itemize}
        \item gg start of file; G end of file
        \item 123G go to specific line number
        \end{itemize}
    \subsection{By marker:}
        \begin{itemize}
        \item mx set mark x; 'x go to mark x
        \item '. go to position of last edit
        \item ' ' go back to last point before jump
        \end{itemize}
    \subsection{return to last position}
        \textasciicircum o to return to last location

    \subsection{Scrolling:}
        \begin{itemize}
        \item \textasciicircum F forward full screen; \textasciicircum B backward full screen
        \item \textasciicircum D down half screen; \textasciicircum U up half screen
        \item \textasciicircum E scroll one line up; \textasciicircum Y scroll one line down
        \item zz centre cursor line
    \end{itemize}
\section{Text Objects}
In Vim, editing commands have the following structure:\\
\begin{verbatim}
  <number><command><text object or motion>
  \end{verbatim}
  Objects will be listed here, commands in their own sections.
\begin{itemize}
    \item aw/iw (a word/inner word)
    \item as/is (a sentence/inner sentence)
    \item ap/ip (a paragraph/inner paragraph)
    \item a"/i" (a double quoted string/inside double quoted string)
    \item a'/i'
    \item a`/i`
    \item a(/i(
    \item a\{/i\{
    \item a[/i[
\end{itemize}
\subsection{vim-indent-object}
Text Objects from the package vim-indent-object:
\begin{itemize}
        \item ai/ii (This indent level and the line above/this indent level
            excluding the line above)
\end{itemize}
\subsection{vim-pythonsense}
\begin{itemize}
        
    \item "ac" : Outer class text object. This includes the entire class, including the header (class name declaration) and decorators, the class body, as well as a blank line if this is given after the class definition.
    \item "ic" : Inner class text object. This is the class body only, thus excluding the class name declaration line and any blank lines after the class definition.
    \item "af" : Outer function text object. This includes the entire function, including the header (function name declaration) and decorators, the function body, as well as a blank line if this is given after the function definition.
    \item "if" : Inner function text object. This is the function body only, thus excluding the function name declaration line and any blank lines after the function definition.
    \item "ad" : Outer docstring text object.
    \item "id" : Inner docstring text object.
    \end{itemize}

\section{Editing}

    \begin{itemize}
    \item u undo; \textasciicircum R redo
    \item . repeat last editing command
    \end{itemize}

\section{Inserting}

All insertion commands are terminated with $<$Esc$>$ to return to command mode.
\begin{itemize}
    \item i insert text at cursor; I insert text at start of line
    \item a append text after cursor; A append text after end of line
    \item o open new line below; O open new line above
    \end{itemize}
\section{Changing}

All change commands except r and R are terminated with $<$Esc$>$ to return to command mode.
\begin{itemize}
    \item r replace single character; R replace multiple characters
    \item s change single character
    \item cw change word; C change to end of line; cc change whole line
    \item c$<$motion$>$ changes text in the direction of the motion
    \item ci( change inside parentheses (see text object selection for more examples)
\end{itemize}

\section{Deleting}
\begin{itemize}
    \item x delete char
    \item dw delete word; D delete to end of line; dd delete whole line
    \item d$<$motion$>$ deletes in the direction of the motion
    \end{itemize}

\section{Cut and paste}
\begin{itemize}
    \item yy copy line into paste buffer; dd cut line into paste buffer
    \item p paste buffer below cursor line; P paste buffer above cursor line
    \item xp swap two characters (x to delete one character, then p to put it back after the cursor position)
    \end{itemize}
\section{Commenting}
This is from vim-commentary:\\
Comment stuff out. Use gcc to comment out a line (takes a count), gc to comment
out the target of a motion (for example, gcap to comment out a paragraph), gc in
visual mode to comment out the selection, and gc in operator pending mode to
target a comment. You can also use it as a command, either with a range like
:7,17Commentary, or as part of a :global invocation like with
:g/TODO/Commentary. \\
\subsection{Re-flowing text}
gq reflows text

\section{ArgWrap}
ArgWrap, switches between horizontally and vertically formatted arguments

\section{Blocks}
\begin{itemize}
    \item v visual block stream; V visual block line; \textasciicircum V visual block column\\
        most motion commands extend the block to the new cursor position
        o moves the cursor to the other end of the block
    \item d or x cut block into paste buffer
    \item y copy block into paste buffer
    \item $>$ indent block; $<$ unindent block
    \item gv reselect last visual block
    \end{itemize}

\section{Global}

     :\%s/foo/bar/g substitute all occurrences of "foo" to "bar"\\
        \% is a range that indicates every line in the file\\
        /g is a flag that changes all occurrences on a line instead of just the first one

\section{Searching}
\begin{itemize}

    \item / search forward; ? search backward
    \item * search forward for word under cursor; \# search backward for word under cursor
    \item n next match in same direction; N next match in opposite direction
    \item fx forward to next character x; Fx backward to previous character x
    \item ; move again to same character in same direction; , move again to same character in opposite direction
    \end{itemize}

\section{Files}
\begin{itemize}

    \item :w write file to disk
    \item :w name write file to disk as name
    \item ZZ write file to disk and quit
    \item :n edit a new file; :n! edit a new file without saving current changes
    \item :q quit editing a file; :q! quit editing without saving changes
    \item :e edit same file again (if changed outside vim)
    \item :e . directory explorer
    \end{itemize}

\section{Windows}
\begin{itemize}
    \item \textasciicircum Wn new window
    \item \textasciicircum Wj down to next window; \textasciicircum Wk up to previous window
    \item \textasciicircum W\_ maximise current window; \textasciicircum W= make all windows equal size
    \item \textasciicircum W+ increase window size; \textasciicircum W- decrease window size
    \end{itemize}

\section{Source Navigation}
\begin{itemize}
    \item \% jump to matching parenthesis/bracket/brace, or language block if language module loaded
    \item gd go to definition of local symbol under cursor; \textasciicircum O return to previous position
    \item \textasciicircum ] jump to definition of global symbol (requires tags file); \textasciicircum T return to previous position (arbitrary stack of positions maintained)
    \item \textasciicircum N (in insert mode) automatic word completion
    \end{itemize}

    \section{Macros}
    To enter a macro, type:\\
\begin{verbatim}
q<letter><commands>q
\end{verbatim}

To execute the macro $<$number$>$ times (once by default), type:
\begin{verbatim}
<number>@<letter>
\end{verbatim}

\section{YouCompleteMe}
\begin{itemize}
    \item GoTo
    \item GoToReferences
    \item GetDoc
\end{itemize}


\end{document}

